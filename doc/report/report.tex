% Copyright (C) 2020 Diogo Rodrigues, Breno Pimentel
% Distributed under the terms of the GNU General Public License, version 3

\documentclass[a4paper, 11pt]{report}
\usepackage[top=35mm,bottom=35mm,left=25mm,right=25mm]{geometry} % Margins

% Change section numbers
\renewcommand{\thesection}{\arabic{section}}

% Second page
\usepackage{secondpage}
\usepackage{datetime}

% Appendix
\usepackage{appendix}

% Landscape pages
\usepackage{pdflscape}
\usepackage{multicol}
\setlength\columnsep{50pt}

% Decent underlines
\usepackage[normalem]{ulem}

% Hyper-references
\usepackage{hyperref}

% Imports
\usepackage{import}

% Graphics and images
\usepackage{graphicx} \graphicspath{{./images/}}
\usepackage{tikz}
\usepackage{tikz-qtree}
\usetikzlibrary{automata, positioning, arrows}
\usepackage[justification=centering,font=small,skip=0.5em]{caption}
\usepackage{subcaption}
\usepackage{float}

\usepackage[binary-units=true]{siunitx} %SI units
\usepackage{pgfplots}
\pgfplotsset{compat=newest} % Allows to place the legend below plot
\usepgfplotslibrary{units} % Allows to enter the units nicely
\sisetup{
  round-mode          = places,
  round-precision     = 2,
}

% Encodings (to render letters with diacritics and special characters)
\usepackage[utf8]{inputenc}

% Language
\usepackage[english]{babel}

% Source code and algorithms
%\usepackage{amsmath}
\usepackage{algorithm}
\usepackage[noend]{algpseudocode}
\usepackage{listings}
\lstset{
	basicstyle=\linespread{0.85}\ttfamily,
	basewidth  = {0.50em,1em},
    frame=tbr, % draw frame at top and bottom of the code
    tabsize=4, % tab space width
    numbers=left, % display line numbers on the left
	showstringspaces=false, % don't mark spaces in strings    
    commentstyle=\color{green}, % comment color
    keywordstyle=\color{blue}, % keyword color
	stringstyle=\color{red} % string color
}
\lstdefinelanguage{cisco}{
    keywords = {
        enable,
        copy, reload,
        configure, terminal, end, interface, switchport, mode,
        access, vlan, show, interfaces, ip, address,
        no, shutdown, exit, nat, inside, outside, route, list, permit,
        pool, source, ovrld, overload, running, id
    },
    morecomment = [l]{\#}
}


% Tables with bold rows
\usepackage{tabularx}
\newcommand\setrow[1]{\gdef\rowmac{#1}#1\ignorespaces}
\newcommand\clearrow{\global\let\rowmac\relax}
\clearrow
\usepackage{multirow}

% Tables with vertical center alignment
\usepackage{array}

% Lists and items
\usepackage{enumitem}

% Math stuff
\usepackage[mathscr]{euscript}
\usepackage{amssymb, latexsym} %Load math symbols like \blacksquare, but also load normal \leadsto arrows
\usepackage{mathtools} % For \text{...}
% \usepackage{enumitem}
% \usepackage{xcolor}
\newcommand{\expnumber}[2]{{#1}\mathrm{e}{#2}} % scientific notation
\newcommand{\degree}{^{\circ}}
\newcommand*\xor{\oplus}
\newcommand\expected[1]{\mathbf{E}[#1]}

% Headers and footers
\usepackage{fancyhdr}
\pagestyle{fancyplain}
\fancyhf{}
\lhead{\fancyplain{}{Configuration of a computer network — Report (RCOM 2020/21)}}
\rhead{\fancyplain{}{Class 2, group 4}}
\lfoot{\fancyplain{}{\leftmark}}
\rfoot{\thepage}

% Email
\newcommand{\email}[1]{
{\texttt{\href{mailto:#1}{#1}} }
}

% Metadata
\title{\Huge Configuration and study of a \\ computer network \\ \vspace*{12pt} \Large Report \\ \vspace*{4pt} \large FEUP - RCOM 2020/21}
\author{
Class 2, group 4 \vspace{0.5em} \\
\begin{tabular}{r l}
	\email{up201800170@fe.up.pt} & Breno Accioly de Barros Pimentel \\
	\email{up201806429@fe.up.pt} & Diogo Miguel Ferreira Rodrigues  \\
\end{tabular}
}
\date{23rd of December, 2020}

% Document
\begin{document}
\maketitle
\begin{secondpage}
    Copyright \copyright 2020--\the\year\ Diogo Rodrigues, Breno Pimentel\par
    \IfFileExists{VERSION}{Version \input{VERSION}}{Draft version}\par
    \immediate\write18{./get-commit-info.sh > COMMIT.tex}
    Built on \today~\currenttime~from \href{https://github.com/dmfrodrigues/feup-rcom-l2}{dmfrodrigues/feup-rcom-l2}, commit \input{COMMIT}\unskip.\par
    Permission is granted to copy and distribute this document under the terms of the
    \href{https://creativecommons.org/licenses/by-nc-nd/4.0/}{Creative Commons Attribution-NonCommercial-NoDerivatives 4.0 International}
    public license.
\end{secondpage}
\clearpage

\pagenumbering{arabic}

\section*{Summary}

This project was elaborated as the second project in the context of the curricular unit Computer Networks (RCOM), part of the Integrated Master in Informatics and Computing Engineering (MIEIC) at the Faculty of Engineering of the University of Porto (FEUP).
It concerns the configuration of a computer network, and its test and study through a developed FTP client.

All objectives were fulfilled, as we successfully implemented in C a simple FTP client for file retrieval over the Internet, and the configured network abided to the requirements in the guidelines.

\section*{Introduction} \label{sec:Introduction}

The present project aims at developing a simple FTP client which is to be tested over a configured network.
The resulting source code was developed in the C language, targeting Linux devices.
The network that was configured is composed of three computers, a Cisco Catalyst 3560 Series switch and a Cisco 2900 Series router; the switch will be used to configure two virtual sub-networks, and the router will provide connectivity to the Internet.

This report has the purpose of guiding the reader through the process used to fulfill the project's goals. As such, this report is divided into two parts.
In part 1 we address the design, development and testing of the FTP client.
In part 2 we describe the steps on configuring the computer network, over the course of seven short experiments that build incrementally to finally arrive at the goal network configuration.
This project was tested in the computers of FEUP, room I321, bench 3, with rack computers 2, 3 and 4 and the available hardware of that rack.

\section{Download application} \label{sec:Part1}
\subsection{Application architecture} \label{sec:Arc}

A download application was created in order to test the network configuration.
The first step of the application is to parse and verify if URL syntax is the one described in \textbf{RFC1738}: \texttt{ ftp://[<user>:<password>@]<host>/<url-path>}. We used a regular expression for this.

Then it finds the IP address with the support of \textbf{gethostbyname} and the application opens a connection using a socket through port 21 to the FTP server. 

After connecting, it makes a login in the server. If no credentials were specified, it will login as anonymous.
Next, it sends a PASV command to the FTP server so it can transfer data in passive mode, if successful, the server will reply
6 bytes, the first 4 bytes is the server IP address, the other two is the port in which the server is listening on. 

After entering passive mode, the download application opens another socket, using the port of the server reply, where the data will be transfered.
Finally, it copies the file from url-path to the current working directory and finish the connection sending the QUIT command and closes the sockets.


\subsection{Report of successful download} \label{sec:Dow}

./download ftp://[<user>:<password>@]<host>/<url-path>

\section{Network configuration and analysis} \label{sec:Part2}

\subsection{Experiment 1} \label{sec:Exp1}
\subsubsection{Network architecture} \label{sec:Arc1}

TODO

\subsubsection{Objectives} \label{sec:Obj1}

This experiment consisted in connecting two computers by configuring its IPs.

\subsubsection{Main configuration commands} \label{sec:Com1}

- Activate eth0 interface:

ifconfig eth0 up

- Configure eth0 with 172.16.30.1 and 24 bits mask in tux33

ifconfig eth0 172.16.30.1/24

- In tux34

ifconfig eth0 172.16.30.254/24


\subsubsection{Logs analysis} \label{sec:Log1}

ARP is used to map an IP address to a MAC address.
Before sending a frame, the computer first needs to know the MAC address of the receiver.
If the emitter does not have an entry with the receiver IP in its ARP table, it will broadcast a ARP packet with the receiver IP and wait for its response.
As seen in the logs, ARP broadcasts a request for the desired IP address 172.16.30.254 (tux34).
Then tux34 identifies itself, sending another ARP packet with its MAC address 00:21:5a:5a:7d:74.

Ping generates ICMP packets.

The MAC and IP addresses of the ping packets are the source addresses MAC: 00:21:5a:61:24:92 IP: 172.16.30.1
and destination addresses MAC: 00:21:5a:7d:74 IP: 172.16.30.254

It is possible to determine if a receiving Ethernet frame is ICMP by checking its IPv4 field and verifying the value 0x01 in the protocol byte.
A ARP frame que be identified with the value 0x806 in the Ethernet frame header.

The length of a receiving frame is in its header.

The loopback interface is used by the computer to send and receive packets to itself.
In the experiment, we could verify loopback packets when the source and destination addresses were equal.
It is important for diagnostics.


\subsection{Experiment 2} \label{sec:Exp2}
\subsubsection{Network architecture} \label{sec:Arc2}

TODO

\subsubsection{Objectives} \label{sec:Obj2}

Implement two virtual LANs in a switch.

\subsubsection{Main configuration commands} \label{sec:Com2}

- Creates a Ethernet VLAN

configure terminal
vlan 30
end

- Creates a connection to the computer port 1 to VLAN 30

configure terminal

interface fastethernet 0/1

switchport mode access

switchport access vlan 30

end

\subsubsection{Logs analysis} \label{sec:Log2}

There are two broadcast domains, 172.16.31.255 and 172.16.30.255.

\subsection{Experiment 3} \label{sec:Exp3}
\subsubsection{Network architecture} \label{sec:Arc3}

TODO

\subsubsection{Objectives} \label{sec:Obj3}

Configure a router in Linux.

\subsubsection{Main configuration commands} \label{sec:Com3}

- Configure tuxy4.eth1

ifconfig eth1 up

ifconfig eth1 172.16.31.253/24

- Add port 4

configure terminal

interface fastethernet 0/4

switchport mode access

switchport access vlan 31

end

- Enable IP forwarding

- Disable ICMP echo-ignore-broadcast

\subsubsection{Logs analysis} \label{sec:Log3}

-----

In tux32:

Kernel IP routing table
Destination     Gateway         Genmask         Flags Metric Ref    Use Iface
172.16.30.0     0.0.0.0         255.255.255.0   U     0      0        0 eth0
172.16.31.0     172.16.30.254   255.255.255.0   UG    0      0        0 eth0

-----

In tux33:

Kernel IP routing table
Destination     Gateway         Genmask         Flags Metric Ref    Use Iface
172.16.30.0     0.0.0.0         255.255.255.0   U     0      0        0 eth0
172.16.31.0     172.16.30.254   255.255.255.0   UG    0      0        0 eth0

-----

In tux34:

Kernel IP routing table
Destination     Gateway         Genmask         Flags Metric Ref    Use Iface
172.16.30.0     0.0.0.0         255.255.255.0   U     0      0        0 eth0
172.16.31.0     0.0.0.0         255.255.255.0   U     0      0        0 eth1

-----

A forwarding table entry contains the destination address, a gateway where the data will be redirected, genmask, flags, metric, ref, use and iface.

It is possible to observe in the logs the ARP message requesting for the gateway address 172.16.30.254 when running the ping command in tux33 to tux32.
This happens because this is the next hop in the route, through there it can be redirected to the final destination.

We are able to observe ICMP packets between all computers in the network, meaning all of them are connected.

When running the ping command from tux33 to tux32, the source MAC address is from tux33 but the destination MAC address is from tux34. This also happens because tux34
work as a gateway between both VLANs.

\subsection{Experiment 4} \label{sec:Exp4}
\subsubsection{Network architecture} \label{sec:Arc4}
\subsubsection{Objectives} \label{sec:Obj4}
\subsubsection{Main configuration commands} \label{sec:Com4}
\subsubsection{Logs analysis} \label{sec:Log4}

\subsection{Experiment 5} \label{sec:Exp5}
\subsubsection{Network architecture} \label{sec:Arc5}
\subsubsection{Objectives} \label{sec:Obj5}
\subsubsection{Main configuration commands} \label{sec:Com5}
\subsubsection{Logs analysis} \label{sec:Log5}

\subsection{Experiment 6} \label{sec:Exp6}
\subsubsection{Network architecture} \label{sec:Arc6}
\subsubsection{Objectives} \label{sec:Obj6}
\subsubsection{Main configuration commands} \label{sec:Com6}
\subsubsection{Logs analysis} \label{sec:Log6}

\section*{Conclusion} \label{sec:Conclusion}

\section*{References} \label{sec:References}

\appendix
\appendixpage
\addappheadtotoc
\chapter{Source code}

The source code of this project can be obtained from \href{https://github.com/dmfrodrigues/feup-rcom-l2}{github.com/dmfrodrigues/feup-rcom-l2}.
The source code is made available by \textcopyright~Diogo Rodrigues and Breno Pimentel under the \href{https://www.gnu.org/licenses/gpl-3.0.en.html}{GNU General Public License v3} (GPLv3), which you should have received together with the source code, or that you can otherwise obtain online.

During project development and evaluation the repository remained private, although it can be shared with evaluators on request to clarify the development process or due to other justifiable reasons.
It will be made public once all equivalent curricular unit projects have been evaluated in the present school year.

\newgeometry{top=24mm,bottom=24mm,left=14mm,right=14mm}
\fancyhfoffset{0pt}

\lstinputlisting[basicstyle=\ttfamily\small, caption=\texttt{url\_parser.h}, language=C]{../../download/include/url_parser.h}
\lstinputlisting[basicstyle=\ttfamily\small, caption=\texttt{url\_parser.c}, language=C]{../../download/src/url_parser.c}

\lstinputlisting[basicstyle=\ttfamily\small, caption=\texttt{server\_cmds.h}, language=C]{../../download/include/server_cmds.h}
\lstinputlisting[basicstyle=\ttfamily\small, caption=\texttt{server\_cmds.c}, language=C]{../../download/src/server_cmds.c}

\lstinputlisting[basicstyle=\ttfamily\small, caption=\texttt{download.h}, language=C]{../../download/include/download.h}
\lstinputlisting[basicstyle=\ttfamily\small, caption=\texttt{download.c}, language=C]{../../download/src/download.c}

\restoregeometry

\chapter{Configuration commands}
\lstinputlisting[basicstyle=\ttfamily\small, caption=\texttt{tuxy2\_config.sh}, language=bash]{../../part2/config/tuxy2_config.sh}
\lstinputlisting[basicstyle=\ttfamily\small, caption=\texttt{tuxy3\_config.sh}, language=bash]{../../part2/config/tuxy3_config.sh}
\lstinputlisting[basicstyle=\ttfamily\small, caption=\texttt{tuxy4\_config.sh}, language=bash]{../../part2/config/tuxy4_config.sh}
\lstinputlisting[basicstyle=\ttfamily\small, caption=\texttt{switch.sh}, language=cisco]{../../part2/config/switch.sh}
\lstinputlisting[basicstyle=\ttfamily\small, caption=\texttt{router.sh}, language=cisco]{../../part2/config/router.sh}

\chapter{Logs}

\begin{multicols}{2}
% \lstinputlisting[basicstyle=\linespread{0.85}\ttfamily\tiny, frame=tbr, caption=\texttt{vary-tau.txt}]{../../stats/vary-tau.txt}
% \lstinputlisting[basicstyle=\linespread{0.85}\ttfamily\tiny, frame=tbr, caption=\texttt{vary-fre.txt}]{../../stats/vary-fre.txt}
% \lstinputlisting[basicstyle=\linespread{0.85}\ttfamily\tiny, frame=tbr, caption=\texttt{vary-tau-fre.txt}]{../../stats/vary-tau-fre.txt}
\end{multicols}

\end{document}
